\documentclass[
  % Babel language, also used to load translations
  english,
  % Use A4 paper size, you can change this to eg. letterpaper if you need
  % the letter format. The normal methods to modify the paper size should
  % be picked up by SDAPS automatically.
  % a4paper, % setting this might break the example scan unfortunately
  % letterpaper
  %
  % If you need it, you can add a custom barcode at the center
  %globalid=SDAPS,
  %
  % And the following adds a per sheet barcode at the bottom left
  print_questionnaire_id,
  %
  % You can choose between twoside and oneside. twoside is the default, and
  % requires the document to be printed and scanned in duplex mode.
  oneside,
  %
  % With SDAPS 1.1.6 and newer you can choose the mode used when recognizing
  % checkboxes. valid modes are "checkcorrect" (default), "check" and
  % "fill".
  %checkmode=checkcorrect,
  %
  % The following options make sense so that we can get a better feel for the
  % final look.
  pagemark,
  stamp]{sdapsclassic}

\usepackage{listings}
\usepackage{multicol}

\author{asdf}
\title{blub}

% Test support for repeating the environment.
\ExplSyntaxOn
%\seq_gset_from_clist:Nn \g__sdaps_questionnaire_ids_seq {a,b,c}
\ExplSyntaxOff

\showboxdepth=10
\showboxbreadth=10000
\tracingoutput=1
\tracingonline=1

\begin{document}

  \begin{questionnaire}

\begin{verbatim}
This is a test to check that verbatim works inside the questionnaire environment.
@_öäü*'%asdf
\end{verbatim}

    So, we can just demonstrate the sdapsarray environment quickly:
   \begin{sdapsarray}
      row head &
        col head &
        col head 2 \\
     \vspace*{-\parskip}\vspace{-1em}
     \begin{lstlisting}
This lstlistings is in the row header %&_\^
Note that verbatim would require more code right now.
\end{lstlisting}\vspace{-1em}
      &
        cell 1 &
        cell 2
      \\
      row head3 & cell 3 & cell 4
    \end{sdapsarray}

    Now something similar but flipping it and with rotated headers. The main
    difference is that instead of the verbatim we put a tabular environment into
    the header.
   \begin{sdapsarray}[flip,layouter=rotated,align=testalign]
      row head &
        col head &
        col head 2 \\
     \sdapsnested{\begin{tabular}{cc} a & b \\ c & d \end{tabular}} &
        cell 1 &
        cell 2
      \\
      row head3 & cell 3 & cell 4
    \end{sdapsarray}

    And a third test, we align this to the previous one! So if you have multiple
    choicearray questions you can make sure they all line up neatly (they will
    do so by default).
   \begin{sdapsarray}[align=testalign]
       & a & b \\
     This is a lot of text to show that it will correctly wrap too.
     Note that we can use math $f(x) = y$ and \verb:\verb ?=!_\: for example
       & c & d
    \end{sdapsarray}

    The main disadvantages with this environment are:
    \begin{itemize}
      \item no row/column coloring support
      \item no support for grid lines
      \item no support for column or row spanning (unlikely to ever happen)
    \end{itemize}

    The big advantage is that you can easily create questions in a tabular
    fashion that contain fragile content and are stacked vertically instead of
    horizontally. Especially the second thing is hard to manage as it can be
    rather complicated to keep track of the metadata in that case.

    \section{Backward Compatbility}

    The following commands are implemented on top of the new commands and provide
    seemless backward compatbility to a lot of features of the old class. Some
    things are currently (and some may never) be implemented though. Also, the
    forms will not look identical.

    \textbox*{2cm}{asdf}

    \singlemark[var=testvar]{Hallo}{lower}{upper}

    \begin{choicequestion}[cols=3]{Which of the following OMR tools have you heard about?}
      \choiceitem{SDAPS}
      \choicemulticolitem{2}{ asdf }
      \choiceitem{QueXF}

      % Insert a text field. The freeform box automatically scales horizontally
      % The first parameter is the height of the box. The second parameter
      % is the amount of columns it should span.
      \choiceitemtext{1.2cm}{2}{Other \LaTeX:}
      \choicemulticolitem{2}{Auto Multiple Choice}
    \end{choicequestion}

    % And a more compact way of doing it; similar to markgroup
    \begin{choicegroup}{Which software do you prefere for the following tasks?}
      % We have to add the possible choices at the start.
      \groupaddchoice{\LaTeX}
      \groupaddchoice{LibreOffice}
      \groupaddchoice{Microsoft Word}
      \groupaddchoice{other}

      % After that it is possible to add each question.
      \choiceline{writing letters}
      \choiceline{creating tables}
      \choiceline{typesetting equations}
    \end{choicegroup}

    \textbox{2cm}{asdf}

    \section{New Commands}

    The choicegroup class is in reality pretty much the new choicearray environment
    which is renamed. This environment is based on the sdapsarray environment and
    gets all the features from this. You can now configure a lot of aspects:
    \begin{itemize}
      \item Optional argument of the environment:
        \begin{itemize}
          \item {\bfseries horizontal/vertical} flags for orientation
          \item {\bfseries layouter} any of the predefined sdapsarray layouters ("default" or "rotated" currently)
          \item {\bfseries align} to make different environments have equal column widths
          \item {\bfseries var} to append "\_param" to the generated variable names
          \item {\bfseries text} to specify a text for the metadata
        \end{itemize} 
      \item Optional argument of the choice command:
        \begin{itemize}
          \item {\bfseries var} to append "\_param" to the generated variable name instead of numbering them
          \item {\bfseries text} to specify a text for the metadata and turn on the parser that support fragile content
        \end{itemize} 
      \item Optional argument of the question command:
        \begin{itemize}
          \item {\bfseries var} to append "\_param" to the generated variable name instead of numbering them
          \item {\bfseries text} to specify a text for the metadata and turn on the parser that support fragile content
        \end{itemize} 
    \end{itemize} 

    So the same question as above, just with the {\bfseries vertical} option set and {\bfseries rotated} layouter.

    \begin{choicegroup}[vertical,layouter=rotated]{Which software do you prefere for the following tasks?}
      % We have to add the possible choices at the start.
      \groupaddchoice[text=LaTeX,var=latex]{\LaTeX}
      \groupaddchoice[var=lo]{LibreOffice}
      \groupaddchoice[var=ms]{Microsoft Word}
      \groupaddchoice[var=other,text=other]{other, with a nice table: \begin{tabular}{cc} a & b \\ c & d\end{tabular}}

      % After that it is possible to add each question.
      \choiceline[var=letter]{writing letters}
      \choiceline[var=asdf,text=asdf]{
\begin{lstlisting}
asdf
\end{lstlisting}}
      \choiceline[var=eqn]{typesetting equations}
    \end{choicegroup}

  \newpage

%\twocolumn
\begin{multicols}{2}
   \begin{sdapsarray}[layouter=rotated,align=testalign]
      row head &
        col head &
        col head 2 \\
     row 1 &
        cell 1 &
        cell 2
      \\
     row 2 & cell & cell \\
     row 3 & cell & cell \\
     row 4 & cell & cell \\
     row 5 & cell & cell \\
     row 6 & cell & cell \\
     row 7 & cell & cell \\
     row 8 & cell & cell \\
     row 9 & cell & cell \\
     row 10 & cell & cell \\
     row 11 & cell & cell \\
     row 12 & cell & cell \\
     row 13 & cell & cell \\
     row 14 & cell & cell \\
     row 15 & cell & cell \\
     row 16 & cell & cell \\
     row 17 & cell & cell \\
     row 18 & cell & cell \\
     row 19 & cell & cell \\
     row 20 & cell & cell \\
     row 21 & cell & cell \\
     row 22 & cell & cell \\
     row 23 & cell & cell \\
     row 24 & cell & cell \\
     row 25 & cell & cell \\
     row 26 & cell & cell \\
     row 27 & cell & cell \\
     row 28 & cell & cell \\
     row 29 & cell & cell \\
     row 30 & cell & cell \\
     row 31 & cell & cell \\
     row 32 & cell & cell \\
     row 33 & cell & cell \\
     row 34 & cell & cell \\
     row 35 & cell & cell \\
     row 36 & cell & cell \\
     row 37 & cell & cell \\
     row 38 & cell & cell \\
     row 39 & cell & cell \\
     row 40 & cell & cell \\
     row 41 & cell & cell \\
     row 42 & cell & cell \\
     row 43 & cell & cell \\
     row 44 & cell & cell \\
     row 45 & cell & cell \\
     row 46 & cell & cell \\
     row 47 & cell & cell \\
     row 48 & cell & cell \\
     row 49 & cell & cell \\
     row 50 & cell & cell \\
     row 51 & cell & cell \\
     row 52 & cell & cell \\
     last (53) & cell & cell
    \end{sdapsarray}
blub

\end{multicols}

\newpage

    \begin{rangearray}[other]{A test range question.}
      \range[var=a]{question}{lower}{upper}{other}
      \range[var=b,text=question sdaps,lower=lower sdaps,upper=upper sdaps]{question pdf}{lower pdf}{upper pdf}{}
      \range[var=c,other=other sdaps]{question}{lower3}{upper3}{other pdf}
    \end{rangearray}

    \begin{markgroup}{What do you think about the following aspects of \LaTeX?}
      \markline{equation syntax}{bad}{good}
      \markline{rendered equations}{ugly}{beautiful}
      \markline{ease of use}{hard}{easy}
    \end{markgroup}

    \begin{markgroup}{What do you think about the following aspects of \LaTeX?}
      \markline{equation syntax}{bad}{a}
      \markline{rendered equations}{ugly}{b}
      \markline{ease of use}{hard}{c}
    \end{markgroup}

  \end{questionnaire}

\end{document}

